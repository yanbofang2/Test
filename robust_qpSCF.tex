%%%%%%%%% MASTER -- compiles the 4 sections

\documentclass[11pt,letterpaper]{article}

\usepackage{amsmath}
\usepackage{amssymb}
\usepackage{amsfonts}
\usepackage{amsthm}
\usepackage{bbold}
\usepackage{mathrsfs}
\usepackage{cite}
\usepackage{hyperref}

\usepackage{physics}
\usepackage{braket}
\usepackage{graphicx}
\usepackage{booktabs}
\usepackage{enumitem}
\usepackage{simplewick}
\usepackage{simpler-wick}
\usepackage{caption}
\usepackage{subcaption}
% \usepackage{breqn}


%%%%%%%%%%%%%%%%%%%%%%%%%%%%%%%%%%%%%%%%%%%%%%%%%%%%%%%%%%%%%%%%%%%%%%%%%
\pagestyle{plain}                                                      %%
%%%%%%%%%% EXACT 1in MARGINS %%%%%%%                                   %%
\setlength{\textwidth}{6.5in}     %%                                   %%
\setlength{\oddsidemargin}{0in}   %% (It is recommended that you       %%
\setlength{\evensidemargin}{0in}  %%  not change these parameters,     %%
\setlength{\textheight}{8.5in}    %%  at the risk of having your       %%
\setlength{\topmargin}{0in}       %%  proposal dismissed on the basis  %%
\setlength{\headheight}{0in}      %%  of incorrect formatting!!!)      %%
\setlength{\headsep}{0in}         %%                                   %%
\setlength{\footskip}{.5in}       %%                                   %%
%%%%%%%%%%%%%%%%%%%%%%%%%%%%%%%%%%%%                                   %%
%\setlength{\parskip}{0.5\baselineskip}
%\setlength{\parindent}{0in}
%\renewcommand{\baselinestretch}{1.5}
% \newcommand{\required}[1]{\section*{\hfil #1\hfil}}                    %%
\renewcommand{\refname}{\hfil References Cited\hfil}                   %%
% \renewcommand{\thesubsection}{\arabic{section}}
% \renewcommand{\thesubsubsection}{\thesubsection.\arabic{subsubsection}}

%%%%%%%%%%%%%%%%%%%%%%%%%%%%%%%%%%%%%%%%%%%%%%%%%%%%%%%%%%%%%%%%%%%%%%%%%

\mathchardef\mhyphen="2D

%PUT YOUR MACROS HERE
\newcommand{\vek}[1]{{\mathbf{#1}}}
\newcommand{\x}{{\mathbf{r}}}
% \newcommand{\ket}[1]{| #1 \rangle}
% \newcommand{\bra}[1]{\langle #1 |}
% \newcommand{\braket}[2]{\langle #1 | #2 \rangle}
% \newcommand{\matel}[3]{\langle #1 | #2 | #3 \rangle}
\newcommand{\alp}{\ensuremath{_{\alpha}}}
\newcommand{\cor}{\ensuremath{_{\text{C}}}}
\newcommand{\occ}{\mathrm{occ}}
\newcommand{\vir}{\mathrm{vir}}
\newcommand{\mx}{\mathrm{max}}
\newcommand{\T}{\mathrm{T}}
\newcommand{\saa}{\ensuremath{^{\alpha\alpha}}}
\newcommand{\sab}{\ensuremath{^{\alpha\beta}}}
\newcommand{\sba}{\ensuremath{^{\beta\alpha}}}
\newcommand{\sbb}{\ensuremath{^{\beta\beta}}}
\newcommand{\sss}{\ensuremath{^{\sigma\sigma}}}
\newcommand{\ssS}{\ensuremath{^{\sigma\bar{\sigma}}}}
\newcommand{\sSs}{\ensuremath{^{\bar{\sigma}\sigma}}}
\newcommand{\sssp}{\ensuremath{^{\sigma\sigma'}}}
\newcommand{\sabab}{\ensuremath{^{\alpha\beta\alpha\beta}}}
\newcommand{\sbaba}{\ensuremath{^{\beta\alpha\beta\alpha}}}
\newcommand{\saaaa}{\ensuremath{^{\alpha\alpha\alpha\alpha}}}
\newcommand{\sbbbb}{\ensuremath{^{\beta\beta\beta\beta}}}
\newcommand{\p}{\ensuremath{\bar{p}}}
\newcommand{\q}{\ensuremath{\bar{q}}}
\newcommand{\pf}{\ensuremath{\mathrm{pf}}}

\DeclareSymbolFont{matha}{OML}{txmi}{m}{it}% txfonts
\DeclareMathSymbol{\varv}{\mathord}{matha}{118}


\newtheorem*{remark}{Remark}
\newtheorem{definition}{Definition}
\newtheorem{theorem}{Theorem}
\newtheorem{lemma}[theorem]{Lemma}
\newtheorem{corollary}[theorem]{Corollary}

% \title{\textbf{Algorithm for Solving HFB}}
\title{\textbf{Robust Formulation of Projected Quasiparticle Self-Consistent Field}}
\author{Guo P. Chen, Yanbo Fang}
\date{April 10, 2024\\Modified: April 12, 2024; June 27, 2024; March 5, 2025; May 28, 2025\\
Last modified: \today}


\begin{document}

\maketitle

In PHFB and PHF, the generalized Fock matrix may become ill-defined for the same reason that the
nonorthogonal Wick theorem fails. The same numerical issue also arises when we construct
(generalized) Fock matrices for Jordan--Wigner (JW) transformed systems.
In the existing derivation of the generalized Fock matrix, we first evaluate the PHF(B)
energy using the nonorthogonal Wick's theorem and then express the generalized Fock
matrix as the derivative of the energy with respect to the 1RDM.
Combining this formalism with the robust Wick's theorem \cite{robustwick} is not straightforward.
Here, I propose a new formulation of the generalized Fock matrix that it is compatible with
the robust Wick's theorem. The key idea is to rewrite the elements of
the generalize Fock matrix in terms of expectation values
so that we can apply the robust Wick's theorem at the very end.

\section{Definitions}

The Bogoliubov quasiparticle annihilation operators
$\beta_p$ are defined in terms of
fermionic particle operators $c^\dag_q$ and $c_q$
through a Bogoliubov transformation:
\begin{equation}
  \beta_p
  = \sum_{q=1}^M \left(U^*_{qp} c_q + V^*_{qp} c^\dag_q\right)
\end{equation}
for $1 \leq q \leq M$, where $M$ is the number of spinorbitals.
In this note, I follow Blaizot and Ripka and define
\begin{subequations}
\begin{align}
  \mathscr{B}_p
  &= \beta_p, \\
  \mathscr{B}_{M+p}
  &= \beta^\dag_p,
\end{align}
\end{subequations}
and
\begin{subequations}
\begin{align}
  \mathscr{C}_p
  &= c_p,\\
  \mathscr{C}_{M+p}
  &= c^\dag_p.
\end{align}
\end{subequations}
We may then write the Bogoliubov transformation as
\begin{equation}
  \label{eq:bogoliubov}
  \mathscr{B}_P^\dag = \sum_{Q=1}^{2M} \mathscr{C}_Q^\dag \mathcal{W}_{QP}
\end{equation}
for $1 \leq P \leq 2M$, where
\begin{equation}
  \mathcal{W} =
  \begin{pmatrix}
    U & V^*\\
    V & U^*
  \end{pmatrix}
\end{equation}
is unitary.

Let
\begin{equation}
  \label{eq:hfb}
  \ket{0} = \prod_{p=1}^M \beta_p \ket{-},
\end{equation}
and
\begin{equation}
  \ket{Z} = \exp(\frac{1}{2} \sum_{pq} Z_{pq} \beta^\dag_p \beta^\dag_q) \ket{0},
\end{equation}
where the Thouless matrix $Z$ is antisymmetric.


\section{Generalized Brillouin Condition}

We write the PHFB energy as a function of $Z$:
\begin{equation}
  E(Z) =
  \frac{\braket{Z|H\mathcal{P}|Z}}{\braket{Z|\mathcal{P}|Z}},
\end{equation}
where $\mathcal{P}$ is a symmetry projector.
Stationarity implies
\begin{equation}
  \label{eq:stationarity}
  \left.\frac{\partial E}{\partial Z^*_{pq}}\right|_{Z=0}
  = \frac{\braket{0|\beta_q \beta_p (H - E)\mathcal{P}|0}}{\braket{0|\mathcal{P}|0}}
  = 0,
\end{equation}
where I followed the convention and took the Wirtinger derivative with respect to $Z^*$.
(TODO: Need to explain how the derivative is difined and what the independent variables
are in $Z$. This will explain whether we have $1/2$ or not.)
\begin{equation}
  \label{eq:brillouin}
  \braket{0|\beta_q \beta_p (H - E)\mathcal{P}|0} = 0
\end{equation}
is referred to as the generalized Brillouin condition.

Note that we have assumed that
\begin{equation}
  \braket{0|\mathcal{P}|0} \neq 0.
\end{equation}
This is typically true because $\mathcal{P}$ is written as a linear combination of rotation operators,
even if $\braket{0|\mathcal{R}(g)|0} = 0$ for a certain $g$.
Moreover, for the symmetry operator $\mathcal{O}$ corresponding to $\mathcal{P}$,
we may constrain $\braket{0|\mathcal{O}|0}$ to the correct value to ensure that
$\mathcal{P} \ket{0} \neq 0$.
For JW-transformed Hamiltonians, the JW string does not show up in the denominator
so we don't have to worry about it.

TODO: Need to distinguish the formalism here from the generalized Brillouin condition
by Levy and Berthier; see \cite{Kutzelnigg1979}.

\section{Constructing the Generalized Fock Matrix}

Let us first construct the generalized Fock matrix, denoted by $\tilde{\mathcal{F}}$,
in the $\mathscr{B}$ basis. We start by rewriting $1/2$ times the left-hand-side of
Equation~\eqref{eq:brillouin} as
\begin{align}
  \label{eq:brillouin2}
  \tilde{\mathcal{F}}_{p, M+q}
  = \frac{1}{2} \braket{0|\mathscr{B}^\dag_{M+q} \mathscr{B}_p \tilde{H}|0}
  = \frac{1}{2} \braket{0|\{[\mathscr{B}_p, \tilde{H}], \mathscr{B}^\dag_{M+q}\}|0},
\end{align}
where $1 \leq p, q \leq M$ and
\begin{equation}
  \tilde{H} = \mathcal{P}  (H - E) \mathcal{P}.
\end{equation}
We may now extend Equation~\eqref{eq:brillouin2} to the full matrix:
\begin{equation}
  \tilde{\mathcal{F}}_{PQ}
  = \frac{1}{2} \braket{0|\{[\mathscr{B}_P, \tilde{H}], \mathscr{B}^\dag_Q\}|0},
\end{equation}
for $1 \leq P, Q \leq 2M$. And it is readily verified that $\tilde{\mathcal{F}}$ is Hermitian.

The generalized Brillouin condition is satisfied iff
\begin{equation}
  \tilde{\mathcal{F}}
  = \begin{pmatrix}
    \tilde{F} & \\
              & - \tilde{F}^*
  \end{pmatrix},
\end{equation}
where
\begin{equation}
  \tilde{F}_{pq}
  = \frac{1}{2} \braket{0|\{[\beta_p, \tilde{H}], \beta^\dag_q\}|0}
  = \frac{1}{2} \left(
    \braket{0|\beta_p \tilde{H} \beta^\dag_q|0}
    - \braket{0|\tilde{H}|0}
  \right)
  = \frac{1}{2} \braket{0|\beta_p \tilde{H} \beta^\dag_q|0}.
\end{equation}
Since $\tilde{F}$ is Hermitian, we can always diagonalize it by
\begin{equation}
  \label{eq:diag_Ftilde}
  \tilde{F} = C \epsilon C^\dag,
\end{equation}
where $\epsilon$ is a real diagonal matrix.
The diagonalization is equivalent to the basis transformation
\begin{equation}
  \beta^\dag_p \mapsto \sum_{q=1}^M C_{qp} \beta^\dag_q.
\end{equation}
Note that the Bogoliubov vacuum state $\ket{0}$ defined in Equaiton~\eqref{eq:hfb}
does not change under this basis transformation up to a physically inconsequential
phase factor.
Without loss of generality, we hereafter assume $\tilde{F}$ is already diagonal,
i.e.,
\begin{equation}
  \tilde{\mathcal{F}} 
  = \begin{pmatrix}
    \tilde{F} & \\
              & - \tilde{F}^*
  \end{pmatrix}
  = \begin{pmatrix}
    \epsilon & \\
             & -\epsilon
  \end{pmatrix}.
\end{equation}

We now transform the generalized Fock matrix back to the $\mathscr{C}$ basis using
Equation~\eqref{eq:bogoliubov}:
\begin{equation}
  \tilde{\mathcal{F}} = \mathcal{W}^\dag \mathcal{F} \mathcal{W},
\end{equation}
where
\begin{equation}
  \label{eq:gen_fock_phfb}
  \mathcal{F}_{PQ}
  = \frac{1}{2} \braket{0|\{[\mathscr{C}_P, \tilde{H}], \mathscr{C}^\dag_Q\}|0},
\end{equation}
or equivalently,
\begin{equation}
  \label{eq:gen_fock}
  \mathcal{F}
  = \begin{pmatrix}
    F & \Delta \\
    -\Delta^* & - F^*
  \end{pmatrix},
\end{equation}
where
\begin{subequations}
\label{eq:gen_fock2}
\begin{align}
  \label{eq:gen_fock_phf}
  F_{pq}
  &= \frac{1}{2} \braket{0|\{[c_p, \tilde{H}], c^\dag_q\}|0},\\
  \Delta_{pq}
  &= \frac{1}{2} \braket{0|\{[c_p, \tilde{H}], c_q\}|0}.
\end{align}
\end{subequations}
Again, one can readily show that $F$ is Hermitian and $\Delta$ is antisymmetric.
$\mathcal{F}$ is the generalized Fock matrix that we have been looking for.

(TODO: Better write
\begin{equation}
  \mathcal{F}_{PQ}
  = \frac{1}{2} \braket{0|\{[\mathscr{C}_P, \tilde{H}], \mathscr{C}^\dag_Q\}|0}
\end{equation}
and
\begin{subequations}
\begin{align}
  F_p^q
  &= \frac{1}{2} \braket{0|\{[c_p, \tilde{H}], c^\dag_q\}|0},\\
  \Delta_{pq}
  &= \frac{1}{2} \braket{0|\{[c_p, \tilde{H}], c_q\}|0}.
\end{align}
\end{subequations}
)

We see that the generalized Brillouin condition is satisfied once we diagonalize $\mathcal{F}$
via the PHFB SCF equation,
\begin{equation}
  \label{eq:phfb_scf}
  \mathcal{F} \mathcal{W} = \mathcal{W} \tilde{\mathcal{F}}
  = \mathcal{W}
  \begin{pmatrix}
    \epsilon & \\
     & - \epsilon
  \end{pmatrix},
\end{equation}
where diagonal elements of $\epsilon$ can be identified as quasiparticle energies.
Equation~\eqref{eq:phfb_scf} reduces to the HFB SCF equation in the absence of
the projection operator. Note also that
we have the freedom to choose the $C$ matrix in Equation~\eqref{eq:diag_Ftilde},
which corresponds to the gauge invariance of the Bogoliubov amplitude $\mathcal{W}$. 
This is consistent with the Bloch--Messiah decomposition of $\mathcal{W}$.

The advantage of Equation~\eqref{eq:gen_fock}, or Equation~\eqref{eq:gen_fock2},
is that it is compatible with  the robust Wick's theorem.
To see this, we expand the commutator and anticommutator:
\begin{subequations}
\label{eq:gen_fock3}
\begin{align}
  \mathcal{F}_{PQ}
  \nonumber
  &= \frac{1}{2} \braket{0|\mathscr{C}_P (H - E) \mathcal{P} \mathscr{C}^\dag_Q|0}
  - \frac{1}{2} \braket{0|\mathscr{C}^\dag_Q (H - E) \mathcal{P} \mathscr{C}_P|0}\\
  &\quad
  + \frac{1}{2} \braket{0|\mathscr{C}^\dag_Q \mathscr{C}_P (H - E) \mathcal{P}|0}
  - \frac{1}{2} \braket{0|(H - E) \mathcal{P} \mathscr{C}_P \mathscr{C}^\dag_Q|0}\\
  \nonumber
  &= \frac{1}{2} \braket{0|\mathscr{C}_P (H - E) \mathcal{P} \mathscr{C}^\dag_Q|0}
  - \frac{1}{2} \braket{0|\mathscr{C}^\dag_Q (H - E) \mathcal{P} \mathscr{C}_P|0}\\
  &\quad
  + \frac{1}{2} \braket{0|\mathscr{C}^\dag_Q \mathscr{C}_P (H - E) \mathcal{P}|0}
  + \frac{1}{2} \braket{0|(H - E) \mathcal{P} \mathscr{C}^\dag_Q \mathscr{C}_P|0},
\end{align}
\end{subequations}
where I have used Brillouin condition.
Explicitly, we have
\begin{subequations}
\begin{align}
  \label{eq:fock_def}
  F_{pq}
  \nonumber
  &= \frac{1}{2} \braket{0|c_p (H - E) \mathcal{P} c^\dag_q|0}
  - \frac{1}{2} \braket{0|c^\dag_q (H - E) \mathcal{P} c_p|0}\\
  &\quad+ \frac{1}{2} \left(
    \braket{0|c^\dag_q c_p (H - E) \mathcal{P}|0}
    + h.c.
  \right)\\
  %
  \Delta_{pq}
  \nonumber
  &= \frac{1}{2} \braket{0|c_p (H - E) \mathcal{P} c_q|0}
  - \frac{1}{2} \braket{0|c_q (H - E) \mathcal{P} c_p|0}\\
  &\quad + \frac{1}{2} \braket{0|c_q c_p (H - E) \mathcal{P}|0}
  + \frac{1}{2} \braket{0|(H - E) \mathcal{P} c_q c_p|0}.
\end{align}
\end{subequations}

After we write $\mathcal{P}$ as linear combination of rotation operators, each term in
the above equation can be evaluated using the robust Wick's theorem, avoiding potential
division-by-zero issues.
 
A few remarks are in order:
\begin{enumerate}
  \item Since the projection operator does not (anti)commute with $\mathscr{C}_P$ ($\mathscr{C}_Q^\dag$),
    we need to evaluate third-order reduced density matrices (3RDMs) to compute $\mathcal{F}$,
    as opposed to the HFB case where only 2RDMs are needed.
    This makes the problem ``more nonlinear,'' which means (i) we should do DIIS for the generalized
    Fock matrix instead of the 1RDMs (The two are the same in HFB),
    (ii) we should probably only turn on DIIS when we are very close to convergence,
    and (iii) the difference density matrix algorithm for direct Fock matrix build (in AO basis)
    becomes more complicated.

    Note, however, there is already some nonlinearity in KS as opposed to HF.

    Tom thinks we should still do DIIS for the 1RDMs. This should be numerically tested.
    See also Section II.D in the augmented Roothaan--Hall (ARH) paper: \url{https://doi.org/10.1063/1.2974099}.
  \item Note that Equation~\eqref{eq:gen_fock3} contains the PHFB energy $E$. This is
    a consequence of PHFB being a nonorthogonal configuration interaction (NOCI) method.
    I think this is fine, even though it again makes the PHFB SCF equation ``more nonlinear.''
  \item The computational scaling is the same as that of HFB, despite a larger prefactor.
    Naively, the computational cost for building $\mathcal{F}$ equals four times
    (twice after symmetry considerations) that of
    a Thouless gradient calculation, but many intermediate quantities can be reused, so
    the actual cost will be less; see later remarks.

    Whether the SCF is faster than direct minimization or not depends on how fast the
    convergence is in each approach. An advantage of the SCF is that we can combine it
    with DIIS.
  \item This formalism also applies to a transformed Hamiltonian with JW strings.
    To see this, we just need to consider $\tilde{H}$ as the JW-transformed Hamiltonian.
    $\tilde{H}$ is still self-adjoint by the properties of the JW transformation.
    If we then move all the strings to the right and write
    \begin{equation}
      \tilde{H} = \tilde{\tilde{H}} \phi,
    \end{equation}
    where $\phi$ is the JW strings, we can readily evaluate the generalized Fock matrix
    using the robust Wick theorem.
    We can also do PHFB in the presence of JW strings.

    For 2D spin systems, extended JW is necessary. In this case, $\phi$ will be transformed
    into a more general Thouless rotation, but it is still manageable; see our
    first strong--weak duality paper.
  \item This formalism can also be easily implemented in a Fock-space full CI code.
  \item Gus: Similar strategy for constructing the generalized Fock matrix
    has been used for Brueckner CC \cite{Gus1994Brueckner, Gus1995Brueckner}.
  \item In \cite{robustwick}, I presented the robust Wick's theorem assuming that
    the operator string, for which
    we want to evaluate the matrix element, is in normal order with respect to the physical vacuum.
    In the notation of the paper, this means
    \begin{equation}
      \Gamma^{(-)} = 0.
    \end{equation}
    However, this assumption can be lifted. Specifically, we can define
    \begin{equation}
      \label{eq:refactorize}
      \tilde{\tilde{\Gamma}}^r
      = \tilde{\Gamma}^r + \delta_{r,1}\, s_1 \Gamma^{(-)},
    \end{equation}
    where $s_1$ is the largest canonical value (singular value in the PHF case) of the overlap matrix.
    Consequently,
    \begin{equation}
      \Gamma = \sum_r s_r^{-1} \tilde{\tilde{\Gamma}}^r,
    \end{equation}
    and all the results in \cite{robustwick} still apply.

    More specifically, we can easily verify that
    \begin{equation}
      \Gamma^{(-)} = \mathcal{C}^\dag \mathcal{K}^{(-)} \mathcal{C}^*,
    \end{equation}
    where
    \begin{equation}
      \mathcal{K}^{(-)}
      = \begin{pmatrix}
        0 & I\\
        -I & 0
      \end{pmatrix}.
    \end{equation}
    Define
    \begin{equation}
      \tilde{\tilde{\mathcal{K}}}^r
      = \tilde{K}^r + \delta_{r, 1} s_1 \mathcal{K}^{(-)}.
    \end{equation}
    Then we have
    \begin{equation}
      \tilde{\tilde{\Gamma}}^r
      = \mathcal{C}^\dag \tilde{\tilde{\mathcal{K}}}^r \mathcal{C}^*.
    \end{equation}
  
  \item Note also that we can use the structure of $\mathcal{F}$ to cut down the number
    of matrix element evaluations by a factor of $1/4$.

    Specifically, we can construct the full $\mathcal{F}$ by computing the following matrix elements only:
    \begin{itemize}
      \item $\braket{0|c^\dag_q c_p (H - E) \mathcal{P}|0}$,
        $\braket{0|c_p (H - E) \mathcal{P} c_q|0}$;
      \item $\braket{0|c_p (H - E) \mathcal{P} c^\dag_q|0}$,
        $\braket{0|c^\dag_q (H - E) \mathcal{P} c_p|0}$ for $p \geq q$;
      \item $\braket{0|c_q c_p (H - E) \mathcal{P}|0}$,
        $\braket{0|(H - E) \mathcal{P} c_q c_p|0}$ for $p > q$.
    \end{itemize}

    Note that instead of computing $\braket{0|(H - E) \mathcal{P} c_q c_p|0}$,
    it would be better to compute 
    \begin{equation}
      \braket{0|c^\dag_p c^\dag_q (H - E) \mathcal{P}|0}
      = \braket{0|(H - E) \mathcal{P} c_q c_p|0}^*
    \end{equation}
    to avoid building
    \begin{equation*}
      \frac{\braket{0|c_q(g) c_q(g)|g}}{\braket{0|g}}
    \end{equation*}
    in our implementation.
    
  \item Integral direct algorithm should work perfectly since in each term there is always a $\rho(g)$
    that contracts with the two-electron integrals. And we know from experience that direct- and
    exchange-type contractions should be screened differently.
\end{enumerate}

\section{Projected Hartree--Fock Case}

The $F$ matrix defined in Equation~\eqref{eq:gen_fock_phf} reduces to the generalized Fock matrix
in the PHF case. To see this more clearly, we can start from the generalized Brillouin condition
for PHF:
\begin{equation}
  \braket{0|a^\dag_a a_i (H - E) \mathcal{P}|0} = 0,
\end{equation}
where
\begin{equation}
  a^\dag_p = \sum_{q=1}^M c^\dag_q D_{qp}
\end{equation}
for $1 \leq p \leq M$, and
\begin{equation}
  \ket{0} = \prod_{i=1}^{N} a^\dag_i \ket{-}.
\end{equation}
Here, $D$ is the MO coefficient matrix and $N$ is the number of electrons.

We then construct
\begin{equation}
  \tilde{\tilde{F}}_{ia}
  = \frac{1}{2} \braket{0|a^\dag_a a_i \tilde{H}|0}
  = \frac{1}{2} \braket{0|\{[a_i, \tilde{H}], a^\dag_a\}|0},
\end{equation}
generalize it to
\begin{subequations}
\label{eq:gen_fock_phf_mo}
\begin{align}
  \tilde{\tilde{F}}_{pq}
  &= \frac{1}{2} \braket{0|\{[a_p, \tilde{H}], a^\dag_q\}|0}\\
  \nonumber
  &= \frac{1}{2} \braket{0|a_p (H - E) \mathcal{P} a^\dag_q|0}
  - \frac{1}{2} \braket{0|a^\dag_q (H - E) \mathcal{P} a_p|0}\\
  &\quad
  + \frac{1}{2} \braket{0|a^\dag_q a_p (H - E) \mathcal{P}|0}
  - \frac{1}{2} \braket{0|(H - E) \mathcal{P} a_p a^\dag_q|0}\\
  \nonumber
  &= \frac{1}{2} \braket{0|a_p (H - E) \mathcal{P} a^\dag_q|0}
  - \frac{1}{2} \braket{0|a^\dag_q (H - E) \mathcal{P} a_p|0}\\
  &\quad+ \frac{1}{2} \left(
    \braket{0|a^\dag_q a_p (H - E) \mathcal{P}|0}
    + h.c.
  \right).
\end{align}
\end{subequations}
The generalized Brillouin condition is satisfied iff $\tilde{\tilde{F}}$
is occupied--occupied and virtual--virtual block diagonal.
Again, without loss of generality, we can assume
$\tilde{\tilde{F}}$ already diagonal, i.e., $\tilde{\tilde{F}} = \epsilon$,
since orbital rotations within occupied--occupied and
virtual--virtual blocks do not affact $\ket{0}$ up to a
physically inconsequential phase factor.

We now define
\begin{subequations}
\begin{align}
  F_{pq}
  &= \frac{1}{2} \braket{0|\{[c_p, \tilde{H}], c^\dag_q\}|0}\\
  \nonumber
  &= \frac{1}{2} \braket{0|c_p (H - E) \mathcal{P} c^\dag_q|0}
  - \frac{1}{2} \braket{0|c^\dag_q (H - E) \mathcal{P} c_p|0}\\
  &\quad+ \frac{1}{2} \left(
    \braket{0|c^\dag_q c_p (H - E) \mathcal{P}|0}
    + h.c.
  \right).
\end{align}
\end{subequations}
which is consistent with Equations~\eqref{eq:gen_fock_phf} and \eqref{eq:fock_def}.
The PHF SCF equation is now written as
\begin{equation}
  F D = D \tilde{\tilde{F}} = D \epsilon.
\end{equation}
Again, this is now compatible with the robust Wick's theorem.
Specifically, using the notation of the robust Wick paper, we now have
\begin{equation}
  \kappa^{01} = \kappa^{10} = 0,
\end{equation}
and correspondingly,
\begin{equation}
  \tilde{\kappa}^{01,r} = \tilde{\kappa}^{10,r} = 0.
\end{equation}
$\tilde{\rho}^{01,r}$ is defined by Equations (25a) and (S39) in \cite{robustwick}.

An added benefit of this formalism is that the occupied--occupied
and virtual--virtual blocks of this generalize Fock matrix do not vanish at convergence,
a problem discussed in the PHF paper \cite{Carlos2012PHF}.
This can be seen from Equation~\eqref{eq:gen_fock_phf_mo}.
In fact, we should probably define
\begin{equation}
  F_{pq} \mapsto
  \frac{F_{pq}}{\braket{0|\mathcal{P}|0}}
\end{equation}
in consistence with Equation~\eqref{eq:stationarity}.
This will make the eigenvalues of $F$ more stable over the SCF iteration,
and we can interpret them as occupied and virtual energies in the sense of
a generalized Koopmans' theorem.

\section{Workding Equations}%
\label{sec:WorkdingEquations}

For the simplicity of notation, we define
\begin{equation}
  \Gamma_{PQ}(g)
  = \frac{\braket{0|\mathscr{C}_P \mathscr{C}_Q|g}}{\braket{0|g}}
\end{equation}
and, in matrix form,
\begin{equation}
  \Gamma(g)
  = \begin{pmatrix}
    \gamma_{\cdot\cdot}(g) &
    \gamma_{\cdot\phantom{\cdot}}^{\phantom{\cdot}\cdot}(g) \\
    \gamma_{\phantom{\cdot}\cdot}^{\cdot\phantom{\cdot}}(g) &
    \gamma^{\cdot\cdot}(g)
  \end{pmatrix},
\end{equation}
where
\begin{subequations}
\begin{align}
  \gamma_{pq}(g)
  &= \frac{\braket{0|c_p c_q|g}}{\braket{0|g}},\\
  \gamma^{p\phantom{q}}_{\phantom{p}q}(g)
  &= \frac{\braket{0|c^\dag_p c_q|g}}{\braket{0|g}},\\
  \gamma^{\phantom{p}q}_{p\phantom{q}}(g)
  &= \frac{\braket{0|c_p c^\dag_q|g}}{\braket{0|g}},\\
  \gamma^{pq}(g)
  &= \frac{\braket{0|c^\dag_p c^\dag_q|g}}{\braket{0|g}}.
\end{align}
\end{subequations}
We see that
\begin{subequations}
\begin{align}
  \gamma_{pq} &= -\gamma_{qp}\\
  \gamma^{pq} &= -\gamma^{qp}.
\end{align}
\end{subequations}
This notation is an extension to that in \cite{Kutzelnigg1997} to the case of operator strings
that are not necessarily normal ordered.

We further define
\begin{equation}
  \bar{\Gamma}_{PQ}(g)
  = \frac{\braket{0|\mathscr{C}_P \mathscr{C}_Q(g)|g}}{\braket{0|g}} 
\end{equation}
and
\begin{subequations}
\begin{align}
  \bar{\gamma}_{pq}(g)
  &= \frac{\braket{0|c_p c_q(g)|g}}{\braket{0|g}},\\
  \bar{\gamma}^{p\phantom{q}}_{\phantom{p}q}(g)
  &= \frac{\braket{0|c^\dag_p c_q(g)|g}}{\braket{0|g}},\\
  \bar{\gamma}^{\phantom{p}q}_{p\phantom{q}}(g)
  &= \frac{\braket{0|c_p c^\dag_q(g)|g}}{\braket{0|g}},\\
  \bar{\gamma}^{pq}(g)
  &= \frac{\braket{0|c^\dag_p c^\dag_q(g)|g}}{\braket{0|g}}.
\end{align}
\end{subequations}

\begin{subequations}
\begin{align}
  h_p^q
  &= \braket{\chi_p|h|\chi_q},\\
  \varv_{pq}^{rs}
  &= \braket{\chi_p \chi_q|V_{\mathrm{ee}}|\chi_r \chi_s}.
\end{align}
and
\begin{equation}
  \bar{\varv}_{pq}^{rs}
  = \varv_{pq}^{rs} - \varv_{pq}^{sr}.
\end{equation}
\end{subequations}
We see that
\begin{equation}
  h_p^q = h_q^{p*},
\end{equation}
\begin{equation}
  \bar{\varv}_{pq}^{rs}
  = \bar{\varv}_{rs}^{pq*},
\end{equation}
and
\begin{equation}
  \bar{\varv}_{pq}^{rs}
  = -\bar{\varv}_{pq}^{sr}
  = -\bar{\varv}_{qp}^{rs}
  = \bar{\varv}_{qp}^{sr}.
\end{equation}

The \textit{ab initio} Hamiltonian is written as
\begin{equation}
  H
  = h_r^s c^\dag_r c_s + \frac{1}{4} \bar{\varv}_{rs}^{tw} c^\dag_r c^\dag_s c_w c_t.
\end{equation}
Given $\ket{0}$, we have
\begin{subequations}
\begin{align}
  E
  &= \frac{\braket{0|H\mathcal{P}|0}}{\braket{0|\mathcal{P}|0}}\\
  &= \xi \int d\mu(g) \braket{0|g} H(g)
\end{align}
\end{subequations}
where
\begin{equation}
  \xi = \frac{1}{\braket{0|\mathcal{P}|0}},
\end{equation}
and
\begin{subequations}
\begin{align}
  H(g)
  &= h_r^s \gamma^{r\phantom{s}}_{\phantom{r}s}(g)
  + \frac{1}{4} \bar{\varv}_{rs}^{tw} \left(
    \gamma^{r\phantom{t}}_{\phantom{r}t}(g) \gamma^{s\phantom{w}}_{\phantom{s}w}(g)
    - \gamma^{r\phantom{w}}_{\phantom{r}w}(g) \gamma^{s\phantom{t}}_{\phantom{s}t}(g)
    + \gamma^{rs}(g) \gamma_{wt}(g)
  \right)\\
  &= h_r^s \gamma^{r\phantom{s}}_{\phantom{r}s}(g)
  + \frac{1}{4} \bar{\varv}_{rs}^{tw} \left(
    2 \gamma^{r\phantom{t}}_{\phantom{r}t}(g) \gamma^{s\phantom{w}}_{\phantom{s}w}(g)
    - \gamma^{rs}(g) \gamma_{tw}(g)
  \right)\\
  &= (h_r^s - \frac{1}{2} \lambda^{s\phantom{r}}_{\phantom{s}r}(g))
  \gamma^{r\phantom{s}}_{\phantom{r}s}(g)
  - \frac{1}{4} \lambda_{rs}(g) \gamma^{rs}(g),
\end{align}
\end{subequations}
where we denote
\begin{subequations}
\begin{align}
  \lambda_{rs}(g)
  &= \bar{\varv}_{rs}^{tw} \gamma_{tw}(g),\\
  \lambda^{t\phantom{s}}_{\phantom{t}s}(g)
  &= \bar{\varv}_{rs}^{tw} \gamma^{r\phantom{w}}_{\phantom{r}w}(g),\\
  \lambda^{tw}(g)
  &= \bar{\varv}_{rs}^{tw} \gamma^{rs}(g).
\end{align}
\end{subequations}
It is readily seen that
\begin{subequations}
\begin{align}
  \lambda_{rs}(g) &= -\lambda_{sr}(g),\\
  \lambda^{tw}(g) &= -\lambda^{wt}(g).
\end{align}
\end{subequations}
Therefore,
\begin{subequations}
\begin{align}
  \nonumber
  \braket{0|c_p (H - E) \mathcal{P} c^\dag_q|0}
  &= \int d\mu(g) \braket{0|g}
  \Bigl(
    \big(H(g) - E\big) \bar{\gamma}_{p\phantom{q}}^{\phantom{p}q}(g)
    + h_r^s \gamma_{p\phantom{r}}^{\phantom{p}r}(g) \bar{\gamma}_{s\phantom{q}}^{\phantom{s}q}(g) 
    - h_r^s \gamma_{ps}(g) \bar{\gamma}^{rq}(g)\\
    \nonumber
    &\hspace{8em}+ 4 \cdot \frac{1}{4} \bar{\varv}_{rs}^{tw}
    \gamma^{s\phantom{w}}_{\phantom{s}w}(g) 
    \gamma_{p\phantom{r}}^{\phantom{p}r}(g) 
    \bar{\gamma}_{t\phantom{q}}^{\phantom{t}q}(g) 
    - 4 \cdot \frac{1}{4} \bar{\varv}_{rs}^{tw}
    \gamma^{s\phantom{w}}_{\phantom{s}w}(g) 
    \gamma_{pt}(g) 
    \bar{\gamma}^{rq}(g)\\
    &\hspace{8em}+ 2 \cdot \frac{1}{4} \bar{\varv}_{rs}^{tw}
    \gamma_{wt}(g)
    \gamma_{p\phantom{r}}^{\phantom{p}r}(g)
    \bar{\gamma}^{sq}(g)
    + 2 \cdot \frac{1}{4} \bar{\varv}_{rs}^{tw}
    \gamma^{rs}(g)
    \gamma_{pw}(g)
    \bar{\gamma}_{t\phantom{q}}^{\phantom{t}q}(g)
  \Bigr)\\
  \nonumber
  &= \int d\mu(g) \braket{0|g}
  \Bigl(
    \big(H(g) - E\big) \bar{\gamma}_{p\phantom{q}}^{\phantom{p}q}(g)
    + h_r^s \gamma_{p\phantom{r}}^{\phantom{p}r}(g) \bar{\gamma}_{s\phantom{q}}^{\phantom{s}q}(g) 
    - h_r^s \gamma_{ps}(g) \bar{\gamma}^{rq}(g)\\
    \nonumber
    &\hspace{8em}- \lambda^{t\phantom{r}}_{\phantom{t}r}(g) 
    \gamma_{p\phantom{r}}^{\phantom{p}r}(g) \bar{\gamma}_{t\phantom{q}}^{\phantom{t}q}(g)
    + \lambda^{t\phantom{r}}_{\phantom{t}r}(g) 
    \gamma_{pt}(g) \bar{\gamma}^{rq}(g)\\
    &\hspace{8em}+ \frac{1}{2} \lambda_{rs}(g)
    \gamma_{p\phantom{s}}^{\phantom{p}s}(g)
    \bar{\gamma}^{rq}(g)
    + \frac{1}{2} \lambda^{tw}(g)
    \gamma_{pw}(g)
    \bar{\gamma}_{t\phantom{q}}^{\phantom{t}q}(g)
  \Bigr)\\
  \nonumber
  &= \int d\mu(g) \braket{0|g}
  \Biggl(
    \big(H(g) - E\big) \bar{\gamma}_{p\phantom{q}}^{\phantom{p}q}(g)\\
    \nonumber
    &\hspace{8em}
    + \left(
      \big(h_r^s - \lambda^{s\phantom{r}}_{\phantom{s}r}(g)\big)
      \gamma_{p\phantom{r}}^{\phantom{p}r}(g)
      - \frac{1}{2} \lambda^{rs}(g) \gamma_{pr}(g)
    \right)
      \bar{\gamma}_{s\phantom{q}}^{\phantom{s}q}(g)\\
    &\hspace{8em}
    - \left(
      \big(h_r^s - \lambda^{s\phantom{r}}_{\phantom{s}r}(g)\big)
      \gamma_{ps}(g)
      - \frac{1}{2} \lambda_{rs}(g) \gamma_{p\phantom{s}}^{\phantom{p}s}(g)
    \right)
    \bar{\gamma}^{rq}(g)
  \Biggr).
\end{align}
\end{subequations}
Note that $\bar{\gamma}_{p\phantom{q}}^{\phantom{p}q}(g)$ is disconnected,
signaling the size-nonextensivity of PHF(B). Yet, in the HF case, this term
vanishes as $\big(H(g) - E\big)$ vanishes, which is consistent with the
size-extensivity of the mean-field approximation.
However, is it possible that the $\bar{\gamma}_{p\phantom{q}}^{\phantom{p}q}(g)$
term also vanishes in the PHF(B) case at convergence?
Unlikely, but we should test it numerically.
If the answer turns out to be yes, can we prove some sort of
linked diagram theorem for PHF(B)?

\begin{subequations}
\begin{align}
  \nonumber
  \braket{0|c^\dag_q (H - E) \mathcal{P} c_p|0}
  &= \int d\mu(g) \braket{0|g}
  \Bigl(
    \big(H(g) - E\big) \bar{\gamma}_{\phantom{q}p}^{q\phantom{p}}(g)
    + h^{s}_{r}\gamma^{qr}(g) \bar\gamma_{sp}(g)-h^{s}_{r}\gamma^{q\phantom{s}}_{\phantom{q}s}(g)\bar\gamma^{r\phantom{p}}_{\phantom{r}p}(g)\\
    \nonumber
    &\hspace{8em}+ 4 \cdot \frac{1}{4}  \bar\varv^{tw}_{rs} \gamma_{\phantom{s}w}^{s\phantom{w}}(g)
        \gamma^{qr}(g)\bar{\gamma}_{tp}(g) 
    + 4 \cdot \frac{1}{4} \bar\varv^{tw}_{rs}\gamma_{\phantom{s}t}^{s\phantom{t}}(g)\gamma_{\phantom{q}w}^{q\phantom{w}}(g)\bar{\gamma}_{\phantom{r}p}^{r\phantom{p}}(g)\\
    &\hspace{8em}- 2 \cdot \frac{1}{4} \bar\varv^{tw}_{rs}\gamma_{wt}(g)\gamma^{qs}(g)\bar{\gamma}_{\phantom{r}p}^{r\phantom{p}}(g)
    - 2 \cdot \frac{1}{4} \bar\varv^{tw}_{rs}\gamma^{rs}(g)\gamma_{\phantom{q}t}^{q\phantom{t}}(g)\bar{\gamma}_{wp}(g)
  \Bigr)\\
  \nonumber
  &= \int d\mu(g) \braket{0|g}
  \Bigl(
    \big(H(g) - E\big) \bar{\gamma}_{\phantom{q}p}^{q\phantom{p}}(g)
    + h^{s}_{r}\gamma^{qr}(g) \bar\gamma_{sp}(g)-h^{s}_{r}\gamma^{q\phantom{s}}_{\phantom{q}s}(g)\bar\gamma^{r\phantom{p}}_{\phantom{r}p}(g)\\
    \nonumber
    &\hspace{8em}- \lambda^{t\phantom{r}}_{\phantom{t}r}(g) 
    \gamma^{qr}(g) \bar{\gamma}_{tp}(g)
    + \lambda^{w\phantom{r}}_{\phantom{w}r}(g) 
    \gamma^{q\phantom{w}}_{\phantom{q}w}(g) \bar{\gamma}^{r\phantom{p}}_{\phantom{r}p}(g)\\
    &\hspace{8em}+ \frac{1}{2} \lambda_{rs}(g)
    \gamma^{qs}(g)
    \bar{\gamma}^{r\phantom{p}}_{\phantom{r}p}(g)
    - \frac{1}{2} \lambda^{tw}(g)
    \gamma^{q\phantom{t}}_{\phantom{q}t}(g)
    \bar{\gamma}_{wp}(g)
  \Bigr)\\
  \nonumber
  &= \int d\mu(g) \braket{0|g}
  \Biggl(
    \big(H(g) - E\big) \bar{\gamma}_{\phantom{q}p}^{q\phantom{p}}(g)\\
    \nonumber
    &\hspace{8em}
    + \left(
      \big(h_r^s - \lambda^{s\phantom{r}}_{\phantom{s}r}(g)\big)
      \gamma^{qr}(g)
      - \frac{1}{2} \lambda^{rs}(g) \gamma^{q\phantom{r}}_{\phantom{q}r}(g)
    \right)
    \bar{\gamma}_{sp}(g)\\
    &\hspace{8em}
    - \left(
      \big(h_r^s - \lambda^{s\phantom{r}}_{\phantom{s}r}(g)\big)
      \gamma_{\phantom{q}s}^{q\phantom{s}}(g)
      + \frac{1}{2} \lambda_{rs}(g) \gamma^{qs}(g)
    \right)
    \bar{\gamma}^{r\phantom{p}}_{\phantom{r}p}(g)
  \Biggr).
\end{align}
\end{subequations}
Define
\begin{align}
  \chi^{qs}(g)
  &= \big(h_r^s - \lambda^{s\phantom{r}}_{\phantom{s}r}(g)\big)
  \gamma^{qr}(g)
  - \frac{1}{2} \lambda^{rs}(g) \gamma^{q\phantom{r}}_{\phantom{q}r}(g),\\
  %
  \chi^{q\phantom{r}}_{\phantom{q}r}(g)
  &= \big(h_r^s - \lambda^{s\phantom{r}}_{\phantom{s}r}(g)\big)
  \gamma_{\phantom{q}s}^{q\phantom{r}}(g)
  + \frac{1}{2} \lambda_{rs}(g) \gamma^{qs}(g),\\
  %
  \chi_{p\phantom{s}}^{\phantom{p}s}(g)
  &= \big(h_r^s - \lambda^{s\phantom{r}}_{\phantom{s}r}(g)\big)
  \gamma_{p\phantom{r}}^{\phantom{p}r}(g)
  - \frac{1}{2} \lambda^{rs}(g) \gamma_{pr}(g),\\
  %
  \chi_{pr}(g)
  &= \big(h_r^s - \lambda^{s\phantom{r}}_{\phantom{s}r}(g)\big)
  \gamma_{ps}(g)
  - \frac{1}{2} \lambda_{rs}(g) \gamma_{p\phantom{s}}^{\phantom{p}s}(g).
\end{align}
We can then write
\begin{subequations}
\begin{align}
  \braket{0|c_p (H - E) \mathcal{P} c^\dag_q|0}
  &= \int d\mu(g) \braket{0|g}
  \Big(
    \big(H(g) - E\big) \bar{\gamma}_{p\phantom{q}}^{\phantom{p}q}(g)
    + \chi_{p\phantom{s}}^{\phantom{p}s}(g) \bar{\gamma}_{s\phantom{q}}^{\phantom{s}q}(g)
    - \chi_{pr}(g) \bar{\gamma}^{rq}(g)
  \Big),\\
  %
  \braket{0|c^\dag_q (H - E) \mathcal{P} c_p|0}
  &= \int d\mu(g) \braket{0|g}
  \Big(
    \big(H(g) - E\big) \bar{\gamma}_{\phantom{q}p}^{q\phantom{p}}(g)
    + \chi^{qs}(g) \bar{\gamma}_{sp}(g)
    - \chi^{q\phantom{r}}_{\phantom{q}r}(g) \bar{\gamma}^{r\phantom{p}}_{\phantom{r}p}(g) 
  \Big).
\end{align}
\end{subequations}

Similarly,
\begin{subequations}
\begin{align}
  \nonumber
  \braket{0|c^\dag_q c_p (H - E)\mathcal{P}|0}
  &= \int d\mu(g) \braket{0|g}
  \Bigl(
    \big(H(g) - E\big) {\gamma}_{\phantom{q}p}^{q\phantom{p}}(g)
    - h^{s}_{r} \gamma^{qr}(g) \gamma_{ps}(g)
    + h^{s}_{r} \gamma^{q\phantom{s}}_{\phantom{q}s}(g) \gamma^{\phantom{p}r}_{p\phantom{r}}(g)\\
    \nonumber
    &\hspace{8em}
    + 4 \cdot \frac{1}{4} \bar\varv^{tw}_{rs}
    \gamma^{s\phantom{t}}_{\phantom{s}t}(g) \gamma^{qr}(g) \gamma_{pw}(g) 
    - 4 \cdot \frac{1}{4} \bar\varv^{tw}_{rs}
    \gamma^{s\phantom{t}}_{\phantom{s}t}(g) \gamma^{q\phantom{w}}_{\phantom{q}w}(g) \gamma_{p\phantom{r}}^{\phantom{p}r}(g)\\
    &\hspace{8em}
    - 2 \cdot \frac{1}{4} \bar\varv^{tw}_{rs}
    \gamma_{wt}(g) \gamma^{\phantom{p}s}_{p\phantom{s}}(g) \gamma^{qr}(g)
    - 2 \cdot \frac{1}{4} \bar\varv^{tw}_{rs}
    \gamma^{rs}(g) \gamma^{q\phantom{w}}_{\phantom{q}w}(g) \gamma_{pt}(g) 
  \Bigr)\\
  \nonumber
  &= \int d\mu(g) \braket{0|g}
  \Bigl(
    \big(H(g) - E\big) {\gamma}_{\phantom{q}p}^{q\phantom{p}}(g)
    -h^{s}_{r}\gamma^{qr}(g)\gamma_{ps}(g)+h^{s}_{r}\gamma^{q\phantom{s}}_{\phantom{q}s}(g)\gamma^{\phantom{p}r}_{p\phantom{r}}(g)\\
    \nonumber
    &\hspace{8em}+ \lambda^{w\phantom{r}}_{\phantom{w}r}(g) 
    \gamma^{qr}(g) \gamma_{pw}(g)
    - \lambda^{w\phantom{r}}_{\phantom{w}r}(g) 
    \gamma_{\phantom{q}w}^{q\phantom{w}}(g) \gamma^{\phantom{p}r}_{p\phantom{r}}(g)\\
    &\hspace{8em}+ \frac{1}{2} \lambda_{rs}(g)
    \gamma_{p\phantom{s}}^{\phantom{p}s}(g)
    \gamma^{qr}(g)
    - \frac{1}{2} \lambda^{tw}(g)
    \gamma^{q\phantom{w}}_{\phantom{q}w}(g)
    \gamma_{pt}(g)
  \Bigr)\\
  \nonumber
  &= \int d\mu(g) \braket{0|g}
  \Biggl(
    \big(H(g) - E\big) {\gamma}_{\phantom{q}p}^{q\phantom{p}}(g)\\
    \nonumber
    &\hspace{8em}
    + \left(
      \big(h_r^s - \lambda^{s\phantom{r}}_{\phantom{s}r}(g)\big)
      \gamma_{p\phantom{r}}^{\phantom{p}r}(g)
      - \frac{1}{2} \lambda^{rs}(g) \gamma_{pr}(g)
    \right)
    \gamma_{\phantom{q}s}^{q\phantom{s}}(g)\\
    &\hspace{8em}
    - \left(
      \big(h_r^s - \lambda^{s\phantom{r}}_{\phantom{s}r}(g)\big)
      \gamma_{ps}(g)
      - \frac{1}{2} \lambda_{rs}(g) \gamma_{p\phantom{s}}^{\phantom{p}s}(g)
    \right)
    \gamma^{qr}(g)
  \Biggr)\\
  &= \int d\mu(g) \braket{0|g}
  \Big(
    \big(H(g) - E\big) {\gamma}_{\phantom{q}p}^{q\phantom{p}}(g)
    + \chi_{p\phantom{s}}^{\phantom{p}s}(g) \gamma_{\phantom{q}s}^{q\phantom{s}}(g)
    - \chi_{pr}(g) \gamma^{qr}(g)
  \Big),
\end{align}
\end{subequations}
%
\begin{subequations}
\begin{align}
  \nonumber
  \braket{0|c_p (H - E) \mathcal{P} c_q|0}
  &= \int d\mu(g) \braket{0|g}
  \Bigl(
    \big(H(g) - E\big) \bar{\gamma}_{pq}(g)
    + h^{s}_{r}\gamma^{\phantom{p}r}_{p\phantom{r}}(g)\bar\gamma_{sq}(g)-h^{s}_{r}\gamma_{ps}(g)\bar\gamma^{r\phantom{q}}_{\phantom{r}q}(g)\\
    \nonumber
    &\hspace{8em}+ 4 \cdot \frac{1}{4}  \bar\varv^{tw}_{rs} \gamma_{\phantom{s}w}^{s\phantom{w}}(g)\gamma_{p\phantom{r}}^{\phantom{p}r}(g)\bar\gamma_{tq}(g) 
    + 4 \cdot \frac{1}{4} \bar\varv^{tw}_{rs}\gamma_{\phantom{s}t}^{s\phantom{t}}(g)\gamma_{pw}(g)\bar\gamma_{\phantom{r}q}^{r\phantom{q}}(g)\\
    &\hspace{8em}+ 2 \cdot \frac{1}{4} \bar\varv^{tw}_{rs}\gamma_{wt}(g)\gamma_{p\phantom{r}}^{\phantom{p}r}(g)\bar\gamma_{\phantom{s}q}^{s\phantom{q}}(g)
    - 2 \cdot \frac{1}{4} \bar\varv^{tw}_{rs}\gamma^{rs}(g)\gamma_{pt}(g)\bar\gamma_{wq}(g)
  \Bigr)\\
  \nonumber
  &= \int d\mu(g) \braket{0|g}
  \Bigl(
    \big(H(g) - E\big) \bar{\gamma}_{pq}(g)
    + h^{s}_{r}\gamma^{\phantom{p}r}_{p\phantom{r}}(g)\bar\gamma_{sq}(g)-h^{s}_{r}\gamma_{ps}(g)\bar\gamma^{r\phantom{q}}_{\phantom{r}q}(g)\\
    \nonumber
    &\hspace{8em}- \lambda^{t\phantom{r}}_{\phantom{t}r}(g) 
    \gamma_{p\phantom{r}}^{\phantom{p}r}(g) \bar{\gamma}_{tq}(g)
    + \lambda^{w\phantom{r}}_{\phantom{w}r}(g) 
    \gamma_{pw}(g) \bar{\gamma}^{r\phantom{q}}_{\phantom{r}q}(g)\\
    &\hspace{8em}- \frac{1}{2} \lambda_{rs}(g)
    \gamma_{p\phantom{r}}^{\phantom{p}r}(g)
    \bar{\gamma}^{s\phantom{q}}_{\phantom{s}q}(g)
    - \frac{1}{2} \lambda^{tw}(g)
    \gamma_{pt}(g)
    \bar{\gamma}_{wq}(g)
  \Bigr)\\
  \nonumber
  &= \int d\mu(g) \braket{0|g}
  \Biggl(
    \big(H(g) - E\big) \bar{\gamma}_{pq}(g)\\
    \nonumber
    &\hspace{8em}
    + \left(
      \big(h_r^s - \lambda^{s\phantom{r}}_{\phantom{s}r}(g)\big)
      \gamma_{p\phantom{r}}^{\phantom{p}r}(g)
      - \frac{1}{2} \lambda^{rs}(g) \gamma_{pr}(g)
    \right)
    \bar{\gamma}_{sq}(g)\\
    &\hspace{8em}
    - \left(
      \big(h_r^s - \lambda^{s\phantom{r}}_{\phantom{s}r}(g)\big)
      \gamma_{ps}(g)
      - \frac{1}{2} \lambda_{rs}(g) \gamma_{p\phantom{s}}^{\phantom{p}s}(g)
    \right)
    \bar{\gamma}^{r\phantom{q}}_{\phantom{r}q}(g)
  \Biggr)\\
  &= \int d\mu(g) \braket{0|g}
  \Big(
    \big(H(g) - E\big) \bar{\gamma}_{pq}(g)
    + \chi_{p\phantom{s}}^{\phantom{p}s}(g) \bar{\gamma}_{sq}(g)
    - \chi_{pr}(g) \bar{\gamma}^{r\phantom{q}}_{\phantom{r}q}(g)
  \Big),
\end{align}
\end{subequations}
%
\begin{subequations}
\begin{align}
  \nonumber
  \braket{0|c_q c_p (H - E) \mathcal{P} |0}
  &= \int d\mu(g) \braket{0|g}
  \Bigl(
    \big(H(g) - E\big) \gamma_{qp}(g)
    + h^{s}_{r} \gamma^{\phantom{p}r}_{p\phantom{r}}(g) \gamma_{qs}(g)
    - h^{s}_{r} \gamma_{ps}(g) \gamma^{\phantom{q}r}_{q\phantom{r}}(g)\\
    \nonumber
    &\hspace{8em}
    - 4 \cdot \frac{1}{4} \bar\varv^{tw}_{rs}
    \gamma_{\phantom{s}t}^{s\phantom{t}}(g)
    \gamma_{p\phantom{r}}^{\phantom{p}r}(g)
    \gamma_{qw}(g)
    + 4 \cdot \frac{1}{4} \bar\varv^{tw}_{rs}
    \gamma_{\phantom{r}w}^{r\phantom{w}}(g)
    \gamma_{pt}(g)
    \gamma_{q\phantom{s}}^{\phantom{q}s}(g)\\
    &\hspace{8em}
    - 2 \cdot \frac{1}{4} \bar\varv^{tw}_{rs}
    \gamma_{wt}(g)
    \gamma_{p\phantom{s}}^{\phantom{p}s}(g) 
    \gamma_{q\phantom{r}}^{\phantom{q}r}(g)
    - 2 \cdot \frac{1}{4} \bar\varv^{tw}_{rs}
    \gamma^{rs}(g)
    \gamma_{pt}(g)
    \gamma_{qw}(g)
  \Bigr)\\
  \nonumber
  &= \int d\mu(g) \braket{0|g}
  \Bigl(
    \big(H(g) - E\big) \gamma_{qp}(g)
    + h^{s}_{r} \gamma^{\phantom{p}r}_{p\phantom{r}}(g) \gamma_{qs}(g)
    - h^{s}_{r} \gamma_{ps}(g) \gamma^{\phantom{q}r}_{q\phantom{r}}(g)\\
    \nonumber
    &\hspace{8em}
    - \lambda^{w\phantom{r}}_{\phantom{w}r}(g) 
    \gamma^{\phantom{p}r}_{p\phantom{r}}(g) \gamma_{qw}(g) 
    + \lambda^{t\phantom{s}}_{\phantom{t}s}(g) 
    \gamma_{pt}(g) \gamma_{q\phantom{s}}^{\phantom{q}s}(g)\\
    &\hspace{8em}+ \frac{1}{2} \lambda_{rs}(g)
    \gamma_{p\phantom{s}}^{\phantom{p}s}(g)
    \gamma_{q\phantom{r}}^{\phantom{q}r}(g)
    - \frac{1}{2} \lambda^{tw}(g)
    \gamma_{pt}(g)
    \gamma_{qw}(g)
  \Bigr)\\
  \nonumber
  &= \int d\mu(g) \braket{0|g}
  \Biggl(
    \big(H(g) - E\big) \gamma_{qp}(g)\\
    \nonumber
    &\hspace{8em}
    + \left(
      \big(h_r^s - \lambda^{s\phantom{r}}_{\phantom{s}r}(g)\big)
      \gamma_{p\phantom{r}}^{\phantom{p}r}(g)
      - \frac{1}{2} \lambda^{rs}(g) \gamma_{pr}(g)
    \right)
    \gamma_{qs}(g)\\
    &\hspace{8em}
    - \left(
      \big(h_r^s - \lambda^{s\phantom{r}}_{\phantom{s}r}(g)\big)
      \gamma_{ps}(g)
      - \frac{1}{2} \lambda_{rs}(g) \gamma_{p\phantom{s}}^{\phantom{p}s}(g)
    \right)
    \gamma_{q\phantom{r}}^{\phantom{q}r}(g)
  \Biggr)\\
  &= \int d\mu(g) \braket{0|g}
  \Big(
    \big(H(g) - E\big) \gamma_{qp}(g)
    + \chi_{p\phantom{s}}^{\phantom{p}s}(g) \gamma_{qs}(g)
    - \chi_{pr}(g) \gamma_{q\phantom{r}}^{\phantom{q}r}(g)
  \Big),
\end{align}
\end{subequations}
%
\begin{subequations}
\begin{align}
  \nonumber
  \braket{0|c^\dag_p c^\dag_q (H - E) \mathcal{P} |0}
  &= \int d\mu(g) \braket{0|g}
  \Bigl(
    \big(H(g) - E\big) \gamma^{pq}(g)
    -h^{s}_{r}\gamma^{pr}(g)\gamma^{q\phantom{s}}_{\phantom{q}s}(g)+h^{s}_{r}\gamma^{p\phantom{s}}_{\phantom{p}s}(g)\gamma^{qr}(g)\\
    \nonumber
    &\hspace{8em}+ 4 \cdot \frac{1}{4}  \bar\varv^{tw}_{rs}\gamma_{\phantom{s}t}^{s\phantom{t}}(g)\gamma^{pr}(g)\gamma_{\phantom{q}w}^{q\phantom{w}}(g) 
    - 4 \cdot \frac{1}{4} \bar\varv^{tw}_{rs}\gamma_{\phantom{s}t}^{s\phantom{t}}(g)\gamma_{\phantom{p}w}^{p\phantom{w}}(g)\gamma^{qr}(g)\\
    &\hspace{8em}- 2 \cdot \frac{1}{4} \bar\varv^{tw}_{rs}\gamma_{wt}(g)\gamma^{pr}(g)\gamma^{qs}(g) 
    + 2 \cdot \frac{1}{4} \bar\varv^{tw}_{rs}\gamma^{rs}(g)\gamma_{\phantom{p}t}^{p\phantom{t}}(g)\gamma_{\phantom{q}w}^{q\phantom{w}}(g)
  \Bigr)\\
  \nonumber
  &= \int d\mu(g) \braket{0|g}
  \Bigl(
    \big(H(g) - E\big) \gamma^{pq}(g)
    -h^{s}_{r}\gamma^{pr}(g)\gamma^{q\phantom{s}}_{\phantom{q}s}(g)+h^{s}_{r}\gamma^{p\phantom{s}}_{\phantom{p}s}(g)\gamma^{qr}(g)\\
    \nonumber
    &\hspace{8em}+ \lambda^{w\phantom{r}}_{\phantom{w}r}(g) 
    \gamma^{pr}(g) \gamma_{\phantom{q}w}^{q\phantom{w}}(g)
    - \lambda^{w\phantom{r}}_{\phantom{w}r}(g) 
    \gamma^{p\phantom{w}}_{\phantom{p}w}(g) \gamma^{qr}(g)\\
    &\hspace{8em}+ \frac{1}{2} \lambda_{rs}(g)
    \gamma^{pr}(g) \gamma^{qs}(g)
    + \frac{1}{2} \lambda^{tw}(g)
    \gamma_{\phantom{q}w}^{q\phantom{w}}(g)
    \gamma_{\phantom{p}t}^{p\phantom{t}}(g)
  \Bigr)\\
  \nonumber
  &= \int d\mu(g) \braket{0|g}
  \Biggl(
    \big(H(g) - E\big) \gamma^{pq}(g)\\
    \nonumber
    &\hspace{8em}
    - \left(
      \big(h_r^s - \lambda^{s\phantom{r}}_{\phantom{s}r}(g)\big)
      \gamma^{q\phantom{s}}_{\phantom{q}s}(g)
      + \frac{1}{2} \lambda_{rs}(g) \gamma^{qs}(g)
    \right)
    \gamma^{pr}(g)\\
    &\hspace{8em}
    + \left(
      \big(h_r^s - \lambda^{s\phantom{r}}_{\phantom{s}r}(g)\big)
      \gamma^{qr}(g)
      - \frac{1}{2} \lambda^{rs}(g) \gamma_{\phantom{q}r}^{q\phantom{r}}(g)
    \right)
    \gamma_{\phantom{p}s}^{p\phantom{s}}(g)
  \Biggr)\\
  &= \int d\mu(g) \braket{0|g}
  \Big(
    \big(H(g) - E\big) \gamma^{pq}(g)
    - \chi^{q\phantom{r}}_{\phantom{q}r}(g) \gamma^{pr}(g)
    + \chi^{qs}(g) \gamma_{\phantom{p}s}^{p\phantom{s}}(g)
  \Big).
\end{align}
\end{subequations}




\bibliographystyle{nsf}
\bibliography{ref}

\end{document}
